%-*-latex-*-
%-*-latex-*-
\newcommand\COURSE{ciss245}
\newcommand\ASSESSMENT{q3001}
\newcommand\ASSESSMENTTYPE{Quiz}
\newcommand\POINTS{\textwhite{xxx/xxx}}

\input{myquizpreamble}
\input{yliow}
\input{\COURSE}
\textwidth=6in

\renewcommand\TITLE{\ASSESSMENTTYPE \ \ASSESSMENT}
\renewcommand\EMAIL{}


\renewcommand\AUTHOR{jdoe5@cougars.ccis.edu} % CHANGE TO YOURS

\begin{document}
\topmattertwo

%------------------------------------------------------------------------------
\nextq
Write a struct for \verb!Martian! so that a \verb!Martian!
variable has
\begin{enumerate}[nosep]
\li \verb!num_heads!: number of heads
\li \verb!num_arms!: number of arms
\li \verb!num_legs!: number of legs
\li \verb!gender!: gender (as a \verb!char!)
\li \verb!max_speed!: maximum speed of travel (as a \verb!double!)
\end{enumerate}
and create struct variable \verb!marvin! with 1 head, 2 arms, 3 legs,
and gender of \verb!'r'! using an initializer list.
\\
\textsc{Answer:}\vspace{-2mm}
\begin{answercode}
// create Martian struct here

// create marvin here
\end{answercode}

%------------------------------------------------------------------------------
\nextq
You are given
\begin{console}[fontsize=\footnotesize]
struct Position
{
    int x, y;
};
struct Alien
{
    Position p;
    int health;
};
\end{console}
Complete the following to create an array \verb!x! of 100 \verb!Alien!s,
all at the given position \verb!p! (see below) and all with \verb!health!
of 1000.
\\
\textsc{Answer:}\vspace{-2mm}
\begin{answercode}
Position p = {10, 20};

// create x here
\end{answercode}

%------------------------------------------------------------------------------
\nextq
An \verb!Alien! struct value has a \verb!health! member variable.
Something is not quite right with the following. Fix it so that it works.
\\
\textsc{Answer:}\vspace{-2mm}
\begin{answercode}
Alien * alien = new Alien;
alien.health = 0; 
\end{answercode}

%------------------------------------------------------------------------------
\nextq
Complete the following (see STEP 1 and STEP 2))
so that the program prints
\begin{console}[fontsize=\footnotesize]
<Weapon type:sword, name:guthwine>
<Person name:eomer, health:1000, weapon:<Weapon type:sword, name:guthwine>>
\end{console}
\\
\textsc{Answer:}\vspace{-2mm}
\begin{answercode}
#include <iostream>

struct Weapon
{
    char type[1024];
    char name[1024];
};

struct Person
{
    char name[1024];
    int health;
    Weapon * weapon;
};

void Weapon_print(Weapon & w)
{
    std::cout << "<Weapon type:" << w.type << ", name:" << w.name << '>';
}

void Weapon_println(Weapon & w)
{
    Weapon_print(w);
    std::cout << '\n';
}

void Person_println(Person & p)
{
    std::cout << "<Person name:" << p.name
              << ", health:" << p.health
              << ", weapon:";
    // STEP 2: Add a statement here to print the weapon of p
    std::cout << ">\n";
}

int main()
{
    Weapon w = {"sword", "guthwine"};
    Weapon_println(w); // prints <Weapon type:sword, name:guthwine>
    
    // STEP 1: Create variable p of type Person and with name of "eomer",
    // health of 1000, and has weapon w (from above).
    Person p = {?, ?, ?};

    // The following should print
    // <Person name:eomer, health:1000, weapon:<Weapon type:sword, name:guthwine>>
    Person_println(p);

    return 0;
}
\end{answercode}


\newpage
\input{instructions.tex}
\end{document}
