%-*-latex-*-
You should know that to find 3-by-3 magic squares you can do this:
enumerate all possible 9 digits numbers and use the 9-digit numbers to 
create 3-by-3 grid and test them to see which 
fits the requirements of being 3-by-3 magix squares.
If you use this to generate the 9-digit numbers:
\begin{console}
for (int n = 0; n <= 999999999; n++)
{
   ...
}
\end{console}
you would have generated 1,000,000,000 grids of 3-by-3 digits.
We can shave off lots of redundant numbers like this:
\begin{console}
for (int n = 123456789; n <= 987654321; n++)
{
   ...
}
\end{console}
But this method of search for magic squares would still go through
lots of numbers which clearly can't be magic squares.
For instance when you \verb!i! reaches
124000000, you will also redundantly go over 124000001, ..., 124009999
which you can tell can't be magic squares since they all have at least two 
zeroes.

One way to speed up the search is to look at one of the requirements for
magic squares: the numbers used must be distinct.
They are called permutations.
Here's an example. 
The following is a permutation of 123:
\[
123, 132, 213, 231, 312, 321
\]
There are only 6 such permutations.

Note that we can build permutations this way:
We start with $[1,2,3]$ as a list of available symbols.
The permutation at this point is empty.
\begin{python}
from latextool_basic import *
def r(x,y,w=3,h=0.6,label='NONE'):
    return Rect(x,y,x+w,y+h,linewidth=0.05,label=r'\texttt{{\small %s}}' % label)
plot = Plot()
plot += r(0,0,label='[[], [1,2,3]]')
print(plot)
\end{python}
We now create permutations of length 1:

\begin{python}
from latextool_basic import *
def r(x,y,w=3,h=0.6,label='NONE'):
    return Rect(x0=x,y0=y,x1=x+w,y1=y+h,linewidth=0.05,label=r'\texttt{\small{%s}}' % label)
plot = Plot()
r0 = r( 0, 0,label='[[], [1,2,3]]')
r1 = r(-4,-1.5,label='[[1], [2,3]]')
r2 = r( 0,-1.5,label='[[2], [1,3]]')
r3 = r( 4,-1.5,label='[[3],[1,2]]')
plot += r0
plot += r1
plot += r2
plot += r3
plot += Line(points=[r0.bottom(),r1.top()])
plot += Line(points=[r0.bottom(),r2.top()])
plot += Line(points=[r0.bottom(),r3.top()])
print(plot)
\end{python}
There are three. 
From the first of the three, I can create two permutations
of length 2:
\begin{python}
from latextool_basic import *
def r(x,y,w=3,h=0.6,label='NONE'):
    return Rect(x0=x,y0=y,x1=x+w,y1=y+h,linewidth=0.05,label=r'\texttt{{\small %s}}' % label)
plot = Plot()
r0 = r( 0, 0,label='[[], [1,2,3]]')
r1 = r(-4,-1.5,label='[[1], [2,3]]')
r2 = r( 0,-1.5,label='[[2], [1,3]]')
r3 = r( 4,-1.5,label='[[3],[1,2]]')
r4 = r( -6, -3, label='[[1,2], [3]]')
r5 = r( -2, -3, label='[[1,3], [2]]')
plot += r0
plot += r1
plot += r2
plot += r3
plot += r4
plot += r5
plot += Line(points=[r0.bottom(),r1.top()])
plot += Line(points=[r0.bottom(),r2.top()])
plot += Line(points=[r0.bottom(),r3.top()])
plot += Line(points=[r1.bottom(),r4.top()])
plot += Line(points=[r1.bottom(),r5.top()])
print(plot)
\end{python}

It should be clear what is happening here.
We have a tree. 
Here are two leaves containing one permutations each:
\begin{python}
from latextool_basic import *
def r(x,y,w=3,h=0.6,label='NONE'):
    return Rect(x0=x,y0=y,x1=x+w,y1=y+h,linewidth=0.05,label=r'\texttt{{\small %s}}' % label)
plot = Plot()
r0 = r( 0, 0,label='[[], [1,2,3]]')
r1 = r(-4,-1.5,label='[[1], [2,3]]')
r2 = r( 0,-1.5,label='[[2], [1,3]]')
r3 = r( 4,-1.5,label='[[3],[1,2]]')
r4 = r( -6, -3, label='[[1,2], [3]]')
r5 = r( -2, -3, label='[[1,3], [2]]')
r6 = r( -6, -4.5, label='[[1,2,3],[]]')
r7 = r( -2, -4.5, label='[[1,3,2],[]]')

plot += r0
plot += r1
plot += r2
plot += r3
plot += r4
plot += r5
plot += r6
plot += r7

plot += Line(points=[r0.bottom(),r1.top()])
plot += Line(points=[r0.bottom(),r2.top()])
plot += Line(points=[r0.bottom(),r3.top()])
plot += Line(points=[r1.bottom(),r4.top()])
plot += Line(points=[r1.bottom(),r5.top()])
plot += Line(points=[r4.bottom(),r6.top()])
plot += Line(points=[r5.bottom(),r7.top()])

print(plot)
\end{python}

Recall that a 3-by-3  magic is a 2D grid of distinct numbers
from 1 to 9 with every row,
every column, and every diagonal adds up to the same value.
You can generate permuations with the above.
The leaves are the permutations.
Therefore all you need to do now is to traverse the tree (say using
inorder traversal) and when the permutation forms a magic square,
you print it (our put is into a container such as a vector).

How big is the tree above?
For the permutation of 3 symbols, the size of the tree is
\[
1 + 3  + 3 \cdot 2 + 3 \cdot 2 \cdot 1
\]
To generate the permutations of 9 symbols (i.e., 1, 2, 3, ..., 9),
the total number of nodes is
\[
1 + 9 + 9 \cdot 8 + 9 \cdot 8 \cdot 7 + \cdots + 9!
\]
This is a huge number. (You can compute that with your calculator.)

But you can do better.
Note that for an $n$-by-$n$ magic square containing
numbers $1, 2, 3, ..., n^2$,
the sum of each row, column, or diagonal is 
\[
\frac{n(n^2 + 1)}{2}
\]
How does this help?
For each node in the tree above, once $n$ numbers are generated,
you have a row and therefore you can can compute the sum of this row.
Once the row does not add up to
\[
\frac{n(n^2 + 1)}{2}
\]
you know that there's no point computing its descendents.
Likewise once you have $2n$ numbers, you can compute the sum of the 
second row.
If the sum is not
\[
\frac{n(n^2 + 1)}{2}
\]
again, you don't have to compute its descendents.
Etc.
Once you have $(n-1)n$ number you can check the first column and 
one of the diagonals.
Here's an example when $n = 3$.
\begin{python}
from latextool_basic import *
def r(x,y,w=4.7,h=0.6,label='NONE',linewidth=0.05):
    return Rect(x0=x,y0=y,x1=x+w,y1=y+h,linewidth=linewidth,label=r'{\footnotesize\texttt{%s}}' % label)
plot = Plot()
r0 = r( 0, 0,label='[[],[1,2,3,4,5,6,7,8,9]]')
r1 = r(-5,-1.5,label='[[1],[2,3,4,5,6,7,8,9]]')
r2 = r( 0,-1.5,label='[[2],[1,3,4,5,6,7,8,9]]')
r3 = r( 5,-1.5,label='...',linewidth=0)
r4 = r( -7, -3, label='[[1,2],[3,4,5,6,7,8,9]]')
r5 = r( -2, -3, label='...', linewidth=0)
r6 = r( -9, -4.5, label='[[1,2,3],[4,5,6,7,8,9]]')
r7 = r( -4, -4.5, label='...', linewidth=0)

plot += r0
plot += r1
plot += r2
plot += r3
plot += r4
plot += r5
plot += r6
plot += r7

plot += Line(points=[r0.bottom(),r1.top()])
plot += Line(points=[r0.bottom(),r2.top()])
plot += Line(points=[r0.bottom(),r3.top()])
plot += Line(points=[r1.bottom(),r4.top()])
plot += Line(points=[r1.bottom(),r5.top()])
plot += Line(points=[r4.bottom(),r6.top()])
plot += Line(points=[r4.bottom(),r7.top()])

print(plot)
\end{python}

Note that the node with the partially completed permutation
\verb![[1,2,3],[4,5,6,7,8,9]]!
need not be expanded further since $1 + 2 + 3$ is not
$3(3^2 + 1)/2 = 15$.
Therefore that is a leaf.


Write a program that accepts $n$ as a command-line argument and,
using the above method, 
\begin{enumerate}[nosep]
\item prints all $n$-by-$n$ magic squares.
\item prints the number of nodes generated 
\end{enumerate}
Each magic square is printed on one line with the numbers in the squares
separate by commas.
Here's a sample run just to fix the output format:
\begin{console}[commandchars=\\\{\}]
g++ main.cpp
./a.out 3
8,1,6,3,5,7,4,9,2
1
\end{console}
The above program discovers one magic square,
prints the magic square and prints 1.
(The above is only to fix output format. There should be more 3-by-3
magic squares.)
Note the command-line argument \texttt{3}
tells the program to print all magic squares of size 3.


\textsc{Note.}
There are other methods to discover magic squares.
You must use the above method since the point is to practice
using trees.

\textsc{Hint.}
\begin{enumerate}[nosep]
\item
  The tree can be build in a depth-first manner using a stack
  (the idea is very similar to depth-first traversal).
  (You can use \verb!stl::list! to simulate a stack if you like.
  STL also comes with a \verb!std::stack! class -- you can use this too.)
  In other words, first create the root pointer pointing to the root node
  allocated in the heap.
  Push the root pointer onto the stack.
  Now in a while loop, as long as the stack is not empty,
  pop a pointer \verb!p! off the stack.
  Check if the node that \verb!p! points to 
  is a leaf. 
  If it's not, create children (on the heap) and attach them to 
  the node that \verb!p! is pointing to.
  Also, push the pointers of these children nodes onto the stack.
\item
  You can print the magic squares as you build the tree 
  or you can traverse the tree after it's completely built,
  printing out leaves that are magic squares.
  
\item
  Each node has two lists of numbers.
  You can for instance use \verb!std::vector< int >! for these two lists of
  numbers.
  In that case you should probably do this:
\begin{Verbatim}[frame=single,fontsize=\footnotesize]
class Data
{
private:
    std::vector< int > permutation;
    std::vector< int > available;
};
\end{Verbatim}
\end{enumerate}



